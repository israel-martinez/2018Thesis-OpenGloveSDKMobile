\glosario{

\begin{itemize}

\item \textbf{Actuadores:} Un actuador es un dispositivo inherentemente mecánico cuya función es proporcionar fuerza para mover o ``actuar” otro dispositivo mecánico. La fuerza que provoca el actuador proviene de tres fuentes posibles: Presión neumática, presión hidráulica, y fuerza motriz
eléctrica (motor eléctrico o solenoide). Dependiendo de el origen de la fuerza el actuador
se denomina ``neumático'',  ``hidráulico'' o ``eléctrico''  \citep{actuadores}. 

\item \textbf{API:} \textit{Application Program Interface} por sus siglas en inglés es código que actúa como interfaz para la programación de aplicaciones, permitiendo por ejemplo, que dos aplicaciones se comuniquen entre si, como el acceder a funcionalidades sin la necesidad de conocer la complejidad del código implementado\citep{techtarget-API}.

\item \textbf{Augmented Reality (AR):} La realidad aumentada (AR) es el uso de información en tiempo real en forma de texto, gráficos, audio y otras mejoras virtuales integradas con objetos del mundo real. Es este elemento del ``mundo real'' lo que diferencia a AR de la realidad virtual. AR integra y agrega valor a la interacción del usuario con el mundo real, frente a una simulación. \citep{gartner-group-AR}.

\item \textbf{Haptic Feedback:} Haptics es una tecnología táctil o de retroalimentación de fuerza que aprovecha el sentido del tacto de una persona al aplicar vibraciones y / o movimiento a la punta del dedo del usuario. Esta estimulación puede ayudar a la tecnología en el desarrollo de objetos virtuales en la pantalla del dispositivo. En su sentido más amplio, hápticos puede ser cualquier sistema que incorpore elementos táctiles y vibre a través de un sentido del tacto \citep{gartner-group-haptics}.

\item \textbf{IMU (Inertial Measurement Unit):} Los sensores inerciales, también llamados IMU (Unidad de medición inercial), son dispositivos electrónicos de medición que permiten estimar la orientación de un cuerpo de las fuerzas inerciales que el cuerpo experimenta. Su principio de funcionamiento se basa en la medición de las fuerzas de aceleración y velocidad angular ejercidas independientemente en masas pequeñas ubicadas en el interior \citep{IMU-sensor-01}.

\item  \textbf{OpenGlove:} es un guante desarrollado por la Universidad de Santiago de Chile, por el grupo de investigación y desarrollo Interaction, el cual provee \textit{haptic feedback} o retroalimentación táctil en ambientes virtuales, como también la captura de movimientos de la mano \citep{openglove-info-page}.

\item \textbf{Mixed reality (MR):}  La realidad mixta es el resultado de mezclar el mundo físico con el mundo digital. La realidad mixta es la siguiente evolución en la interacción entre el hombre, la computadora y el entorno, y abre posibilidades que antes estaban restringidas a nuestra imaginación. Es posible gracias a los avances en visión artificial, potencia de procesamiento gráfico, tecnología de visualización y sistemas de entrada. El término realidad mixta fue presentado originalmente en un artículo de 1994 por Paul Milgram y Fumio Kishino, ``Una taxonomía de visualizaciones de realidad mixta". Su trabajo introdujo el concepto del continuum de virtualidad y se centró en cómo se aplica la categorización de la taxonomía a las exhibiciones. Desde entonces, la aplicación de la realidad mixta va más allá de las pantallas, pero también incluye la información ambiental, el sonido espacial y la ubicación. \citep{microsoft-MR}.

\item \textbf{SDK:} conjunto de utilidades de desarrollo para escribir aplicaciones de software, generalmente asociadas a entornos específicos (por ejemplo, el SDK de Windows) \citep{gartner-group-SDK}.

\item \textbf{UX:}	La ``experiencia de usuario'' abarca todos los aspectos de la interacción del usuario final con la empresa, sus servicios y sus productos \citep{nngroup-ux}.

\item \textbf{Virtual Reality (VR):}  La realidad virtual (VR) proporciona un entorno 3D generado por computadora que rodea al usuario y responde a las acciones de esa persona de forma natural, generalmente a través de pantallas inmersivas montadas en la cabeza y el seguimiento de la cabeza. También se pueden usar guantes que proporcionen seguimiento de las manos y retroalimentación háptica (sensible al tacto). Los sistemas basados en sala brindan una experiencia 3D para múltiples participantes; sin embargo, son más limitados en sus capacidades de interacción. \citep{gartner-group-VR}.

\end{itemize}

}