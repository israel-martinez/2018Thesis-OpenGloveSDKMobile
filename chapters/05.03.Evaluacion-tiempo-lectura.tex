\section{Evaluación tiempo de lectura de datos usando APIs}
El tipo de prueba que se realizó es el mismo que el utilizado por \cite{tesis-cerda-rodrigo}, por tanto se considera como ciclo de lectura cuando el software de control Arduino envía los valores de todos los flexores agregados. Esto es realizado en paralelo a la lectura de todos los datos que entrega el IMU. Cada medición considera el tiempo transcurrido desde la generación del mensaje en el software de control hasta que llega a la API de alto nivel. Se realizaron 1000 pruebas de ciclos lectura utilizando un Baudrate de 57600, un LoopDelay igual a 0 ms y Threshold igual a 0 en la placa Arduino. Se realizaron pruebas con el IMU enviando la información completa (acelerómetro, giroscopio y magnetómetro) y modificando la cantidad de flexores desde uno a 1 a 10, siendo simulados solamente con un flexor físico en la placa. El dispositivo utilizado para la evaluación técnica de las APIs fue el Samsung Galaxy S5 Mini. Se utilizan las mismas aplicaciones desarrolladas para la evaluación de tiempo de activación de actuadores.

\subsection{API C\#}
%Flexores e IMU update resources

% {START} RESUME TABLE ---------------------------------
%\caption[Resumen resultado pruebas de lectura flexores e IMU usando API C\# ]{Resumen resultado pruebas de lectura flexores e IMU usando API C\#  en $\mu s$\\ Fuente: Elaboración propia (2018)}
%\label{table:flexors&imu-xamarin-galaxy-api}

% Table created by stargazer v.5.2.2 by Marek Hlavac, Harvard University. E-mail: hlavac at fas.harvard.edu
% Date and time: Tue, Sep 04, 2018 - 17:29:17
\begin{table}[!htbp] \centering 
\caption[Resumen resultados de pruebas de lectura flexores e IMU usando API C\#]{Resumen resultados de pruebas de lectura flexores e IMU usando API C\# en $\mu s$\\ Fuente: Elaboración propia (2018)}
\label{table:flexors&imu-xamarin-galaxy-api}
\begin{tabular}{@{\extracolsep{5pt}} cccccccc} 
\\[-1.8ex]\hline 
\hline \\[-1.8ex] 
flexors & Mean & Median & Min & Max & Std. Dev. & Skewness & Kurtosis \\ 
 & $\mu s$ & $\mu s$ & $\mu s$ & $\mu s$ & $\mu s$ &     & \\ 
\hline \\[-1.8ex] 
$1$ & $16,038.830$ & $15,068.100$ & $7,417.900$ & $28,865.600$ & $4,484.658$ & $0.692$ & $2.908$ \\ 
$2$ & $1,365.986$ & $1,271.700$ & $376.800$ & $3,635.400$ & $704.429$ & $0.816$ & $3.307$ \\ 
$3$ & $3,133.853$ & $2,734.500$ & $895.200$ & $8,272.900$ & $1,636.889$ & $1.166$ & $3.906$ \\ 
$4$ & $5,251.955$ & $4,796.100$ & $1,605$ & $12,720.200$ & $2,327.944$ & $0.877$ & $3.417$ \\ 
$5$ & $7,391.620$ & $6,823.100$ & $2,496.400$ & $16,335.200$ & $2,972.165$ & $0.739$ & $3.123$ \\ 
$6$ & $9,445.057$ & $8,723.800$ & $3,565$ & $20,047.800$ & $3,703.347$ & $0.815$ & $3.074$ \\ 
$7$ & $11,231.080$ & $10,768.700$ & $4,948.100$ & $21,445.900$ & $3,611.584$ & $0.609$ & $2.891$ \\ 
$8$ & $12,864.060$ & $12,476.600$ & $6,264.900$ & $24,091.800$ & $3,660.714$ & $0.646$ & $3.045$ \\ 
$9$ & $14,512.320$ & $14,120.300$ & $7,632.200$ & $27,282$ & $4,010.673$ & $0.657$ & $2.891$ \\ 
$10$ & $15,967.690$ & $15,187.800$ & $9,808$ & $27,444.800$ & $3,769.420$ & $0.806$ & $3.086$ \\ 
\hline \\[-1.8ex] 
\end{tabular} 
\end{table} 
% {END} RESUME TABLE ---------------------------------

La Figura \ref{fig:xamarin-galaxy-hist-flexor&imu-api}, muestra los histogramas de las latencias obtenidas al recibir los mensajes de los flexores e IMU, modificando la cantidad de flexores de uno a diez.

\begin{figure}
 \begin{center} 
   	\includegraphics[width=1.0\textwidth]{evaluation/graphics/Xamarin/Galaxy-APITest/HistFlexors&IMUXamarinGalaxy-APITest.png}
   \centering
    \caption[Histogramas de Flexores e IMU usando API C\#]{Histogramas de Flexores e IMU usando API C\# \\Fuente: elaboración propia (2018)}
    \label{fig:xamarin-galaxy-hist-flexors&imu-api}
  \end{center}
\end{figure}

\begin{figure}[H]
  \begin{center} 
   	\includegraphics[width=1.0\textwidth]{evaluation/graphics/Xamarin/Galaxy-APITest/NormalQQFlexors&IMUXamarinGalaxy-APITest.png} 
   	\centering
    \caption[Gráfico QQ de Flexores e IMU usando API C\# ]{Flexores e IMU usando API C\# \\Fuente: elaboración propia (2018)} 
    \label{fig:xamarin-galaxy-QQ-flexors&imu-api}
  \end{center}
\end{figure}

\begin{figure}[H]
  \begin{center} 
   	\includegraphics[width=0.8\textwidth]{evaluation/graphics/Xamarin/Galaxy-APITest/BoxplotFlexors&IMUXamarinGalaxy-APITest.png} 
   	\centering
    \caption[Gráficos de cajas de Flexores e IMU usando API C\#  ]{Gráficos de cajas de Flexores e IMU usando API C\# \\Fuente: elaboración propia (2018)} 
    \label{fig:xamarin-galaxy-boxplot-flexors&imu-api}
  \end{center}
\end{figure}

\subsection{API Java}