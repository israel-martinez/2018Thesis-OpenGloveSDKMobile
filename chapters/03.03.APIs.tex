\subsection{APIs}
%según implementación de las APIs en cada prototipo
Para que los desarrolladores puedan hacer uso de las funcionalidades disponibles en el SDK, se requiere del soporte de distintos lenguajes de programación permitiendo abarcar a las más importantes plataformas tecnológicas de Realidad Virtual, Aumentada y Mixta. Por lo tanto se considera la interoperabilidad para los lenguajes C\# y Java, como se especificó en el Requisito no funcional RNF003. Estas APIs fueron desarrolladas paralelamente a la aplicación de configuración pues se hace uso de ella, iniciando con la API de alto nivel de C\#. Una vez que la API C\# fue completada, se replicó utilizando el lenguaje Java. Las mencionadas APIs son clientes WebSockets que establecen conexión con el Servidor WebSocket provisto y administrado por la aplicación de configuración. El primer, segundo y tercer prototipo no incluyen el desarrollo de las APIs de alto nivel, porque estos prototipos abarcaron permitieron evaluar la factibilidad de distintas tecnologías y definir el trabajo a seguir en los siguientes prototipos.

El capítulo Diseño e Implementación especifica en mayor detalle los aspectos arquitecturales  de la solución y del comportamiento en los cuales las APIs están involucradas.

\subsubsection{Cuarto prototipo}
Este prototipo no incluye desarrollo de la APIs de alto nivel, pero si la implementación de distintas tecnologías y la definición del protocolo de mensajes a utilizar por las APIs de los siguientes prototipos. En este prototipo se implementó un servidor WebSocket que recibe conexiones entrantes de clientes WebSocket, los cuales pueden ser la aplicación de configuración y una aplicación de VR por ejemplo. El servidor WebSocket puede activarse o desactivarse desde la aplicación de configuración, permitiendo además definir el el punto de acceso o EndPoint del Servidor. El servidor WebSocket se comunica con la API de Bajo nivel mediante el uso de EventHandlers, los cuales permiten por una parte al Servidor recibir los mensajes provenientes de los dispositivos Bluetooth (datos de flexores e IMU) como también el enviar mensajes para la activación de actuadores, obtención de lista de dispositivos vinculados y la conexión de a los mismos. Es importante destacar que la administración de la conexión de dispositivos Bluetooth está implementada para el Sistema Operativo Android, por lo que es necesario realizar la implementación específica para iOS.
	
\subsubsection{Quinto prototipo}
En este prototipo se  abarcan las funcionalidades referidas a los actuadores, por tanto se aplicó el protocolo de comunicación definido en el cuarto prototipo, entre los clientes y el servidor WebSocket. Con ello se logró cubrir la inicialización, activación y mapeo de los actuadores mediante la API de alto nivel en C\#. Gracias a estas funcionalidades, la aplicación de configuración permite realizar el mapeo y pruebas de activación de los actuadores de los distintos dispositivos OpenGlove conectados.

\subsubsection{Sexto prototipo}
Este prototipo incluye todas las funcionalidades de la API C\# para el uso de OpenGlove en dispositivos móviles utilizando este lenguaje de programación. Esto se logró desarrollando las funcionalidades que no fueron cubiertas en el quinto prototipo, las referidas a los sensores de flexibilidad e IMU. Las funcionalidades implementadas respecto a los sensores de flexibilidad , permiten agregar agregar un flexor a una región para iniciar la transmisión de datos, remover un flexor de una región, calibrar los flexores, testear los flexores asignados a una región y la asignación de un umbral o Threshold (el dispositivo OpenGlove, enviará el dato de un flexor si la diferencia es mayor o igual a este valor). Respecto a las funcionalidades referidas al IMU, se permite iniciar el IMU, asignar el status del IMU, recibir datos crudos o procesados del IMU. Inicio y suspensión de lectura de datos es una funcionalidad que involucra a ambos sensores.