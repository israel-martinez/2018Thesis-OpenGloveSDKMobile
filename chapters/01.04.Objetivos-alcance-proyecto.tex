\section{Objetivos y alcance del proyecto}

\subsection{Objetivo general}

El objetivo general del proyecto es desarrollar un SDK que permita dar soporte a OpenGlove en dispositivos móviles.

%El objetivo general del proyecto es desarrollar una aplicación móvil que permita dar soporte a OpenGlove en dispositivos móviles, como así también desarrollar un SDK que permita la integración con aplicaciones de terceros mediante API.

\subsection{Objetivos específicos}

%Objetivos específicos: Especifican resultados parciales o metas intermedias necesarias para la consecución del objetivo general. Cada objetivo logrado se debe poder evaluar (los resultados son mensurables). Estos objetivos no deben ser confundidos con los propósitos del producto del trabajo. Numerar secuencialmente cada uno de los objetivos para poder identificarlos y así poder referirse a ellos.
En esta sección se listan los resultados parciales definidos para lograr el objetivo general del proyecto de titulación.

\begin{enumerate}
\item Desarrollar la aplicación de configuración de OpenGlove en Android. Permitiendo la creación, visualización, actualización y eliminación de los perfiles de configuración.

\item Desarrollar APIs en Java y C\# que permitan la administración de dispositivos Bluetooth en segundo plano en Android permitiendo la conexión, desconexión, activación y listado de guantes OpenGlove.  También permitirá la activación y control de actuadores \footnote{Definición actuadores: Un actuador es un dispositivo inherentemente mecánico cuya función es proporcionar fuerza para mover o “actuar” otro dispositivo mecánico. Dependiendo de el origen de la fuerza el actuador se denomina ``neumático'', ``hidráulico'' o ``eléctrico'' \citep{actuadores}}, flexores \footnote{  ``... es un sensor de flexión que produce una resistencia variable en función del grado al que este doblada".\citep{flexor-sensor-01}} e IMU (\textit{Inertial Measurement Unit}).

%\item Implementar la autenticación mediante cuentas de Google para acceder a los perfiles de configuración en FireBase \footnote{Base de datos no relacional en la nube Firebase \url{https://firebase.google.com/?hl=es-419}}.

\item Realizar evaluaciones de rendimiento para las APIs Java y C\#.

\item Demostrar el uso del SDK en un ambiente de VR, AR o MR, utilizando OpenGlove y el visor de realidad virtual ZapBox\footnote{Visor Zapbox: \url{https://www.zappar.com/zapbox/}}.
\end{enumerate}


