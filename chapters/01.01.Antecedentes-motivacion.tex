\section{Antecedentes y motivación}
%) Motivación: Esta sección explica el contexto y síntomas de un problema, así como las razones que justificarían resolverlo para alcanzar una situación deseada. Para esto, se describe una situación actual, normalmente indeseable, con datos o hechos relevantes, se identifica a quienes le ocurre, cuando les ocurre, como les ocurre, donde, etc., se identifican los principales síntomas indeseables de la situación. También se explican las consecuencias que tendría en las personas o en la comunidad si la situación actual se mantiene o se la deja evolucionar en forma natural. Normalmente las consecuencias son expresadas en alguna forma de pérdida continua o creciente: humana, material, económica, energética, oportunidades, eficiencia, etc. Se apoya sobre evidencia o referencias apropiadas respaldando las afirmaciones fuertes. Se argumenta sobre la importancia que tiene desarrollar una solución al problema y su impacto, o las consecuencias más relevantes que una solución tendría en las personas, en las empresas, o en alguna disciplina de conocimiento. Se describen finalmente los aspectos más relevantes de una situación deseada.

%Estilo OREO
%\textcolor{red}{AGREGAR OREO}
El tamaño del mercado de la realidad virtual (VR) \footnote{\textbf{Virtual Reality (VR):}  ``La realidad virtual (VR) proporciona un entorno 3D generado por computadora que rodea al usuario y responde a las acciones de esa persona de forma natural ..." \cite{gartner-group-VR}.}  y realidad aumentada (AR)\footnote{\textbf{Augmented Reality (AR):} ``es el uso de información en tiempo real en forma de texto, gráficos, audio y otras mejoras virtuales integradas con objetos del mundo real ... "" \citep{gartner-group-AR}} registra 6.1 billones de dólares para el año 2016 y se estima que para el 2017 ascienda a 11.4 billones \citep{statista-VR-AR}.  En este contexto también aparece la denominada realidad mixta (MR) \footnote{\textbf{Mixed reality (MR):} ``es el resultado de mezclar el mundo físico con el mundo digital,  incluyendo la interacción con objetos virtuales representados en el real ..." \citep{microsoft-MR}.} la cual es un punto intermedio entre VR y AR.  Actualmente es bastante popular el  uso de VR, AR y MR en dispositivos móviles, pero estos carecen del soporte de otros sentidos como lo es el tacto lo cual genera una disrupción cuando se interactúa con los objetos virtuales. En este contexto se puede presenta una brecha cuando cuando se desea incluir retroalimentación vibrotáctil o el denominado \textit{haptic feedback}\footnote{\textbf{Haptic Feedback:} Haptics es una tecnología táctil o de retroalimentación de fuerza que aprovecha el sentido del tacto de una persona al aplicar vibraciones y / o movimiento a la punta del dedo del usuario ..." \citep{gartner-group-haptics}} a aplicaciones en los entornos ya mencionados. Adicionalmente se incluye la captura de movimiento, para su representación en  los ambientes ya mencionados.

%Para los años 2020 y 2021 las estimaciones esperan un tamaño de mercado de 143.3 billones de dólares y 215 billones de dólares 

En primer lugar, el tamaño del mercado entre el 2017 y 2020 para VR y AR espera más de un 1200\% de aumento hablando de billones de dólares. Esto es una interesante oportunidad de negocio, como también el aprovechar las distintas comunidades de desarrollo de VR, AR y MR para masificar el uso de OpenGlove en proyectos de tales ambientes. 

En segundo lugar, la disrupción que se genera cuando se interactúa con objetos virtuales y no se obtiene una respuesta similar a la experiencia real, crea un punto de quiebre entre lo real y lo virtual. Esto implica una inmersión parcial en los ambientes de VR, AR y MR.

En tercer lugar se tiene el alto costo asociados a la adquisición de los guantes. Es posible constatar precios de guantes que van desde los 300\euro \space hasta los 1300\euro \space y licencias desde los 2500\euro \space hasta los 13300\euro . Esto se traduce  costos más altos de desarrollo como también para los usuarios finales. Además genera un mercado con un alcance más limitado.

En base a los argumentos desarrollados, se evidencia una brecha en la inclusión de Haptic Feedback  y la captura de movimiento de las manos en proyectos de VR, AR y MR en dispositivos móviles, la cual actualmente no es cubierta de manera global.

El estado actual de OpenGlove no permite el uso del mismo en comunidades de desarrollo de VR, AR y MR en entornos móviles como Android e iOS. Esto limita la portabilidad y desacople del sistema operativo Windows, reduciendo el alcance que puede tener en las ya mencionadas comunidades de desarrollo.

%END OREO
%END MOTIVACION