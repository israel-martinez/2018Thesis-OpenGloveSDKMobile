\section{Trabajo futuro}

\begin{itemize}
%	\item Redactando ...
	
	\item Dar soporte al SDK en iOS, desarrollando como mínimo la comunicación y administración de dispositivos Bluetooth. Luego desarrollar APIs que soporten otros lenguajes de programación para dispositivos móviles como Kotlin, Swift y Objetive C.
	
	\item Desarrollar patrones de activación de actuadores extensibles en las APIs de alto nivel, para evitar que el desarrollador deba crear sus propios patrones básicos cada vez y que pueda extender estos patrones para crear otros más complejos.
	
%	\item Extender el protocolo de comunicación de las APIs de alto nivel al SDK de Windows, para soportar la activación de actuadores mediante mensajes utilizando WebSocket.
	
	\item Evaluar la posibilidad de extender este proyecto en otras plataformas\footnote{Plataformas soportadas por Xamarin.Forms \url{https://github.com/xamarin/Xamarin.Forms/wiki/Platform-Support}}, dado que Xamarin.Forms soporta las plataformas Android, iOS, UWP (Universal Windows Platform)  y Tizen en una versión estable.
	
	\item Establecer la calibración  del IMU para los diferentes dispositivos OpenGlove para establecer una orientación específica. También desarrollar un algoritmo encargado de interpretar la orientación y posición de la mano (o donde se encuentre el IMU), para obtener una interpretación precisa en tiempo real. Bajo esta misma línea de trabajo, se propone evaluar el uso de los sensores de los dispositivos móviles para establecer una calibración, esto puede ser de uso complementario, ya que al existir diferentes fabricantes, estos no poseen las mismas especificaciones entre cada modelo de smartphone. De manera similar, se podrían utilizar los dispositivos smartwatch, para iOS o Android, utilizando el mismo proyecto Xamarin.Forms para dar soporte a todas las variantes que pueden ser desarrolladas.
	
	\item	 Agregar soporte a otros modelos de sensores de rastreo IMU. Actualmente no es posible seleccionar otro modelo de IMU aparte de SparkFun LSM9DS1, dado que requiere bibliotecas de software específicas en el código de la placa Arduino. Para ello la comunidad debe desarrollar variantes del software de control Arduino para diferentes modelos de IMU. De manera adicional se podría dar soporte una funcionalidad de carga de código en la placa Arduino utilizando la aplicación de configuración. Esto realizarse por ejemplo utilizando la conexión Bluetooth o por medio de conexión USB adecuada. Por ejemplo, la herramienta Bluino Loader - Arduino IDE\footnote{Bluino Loader: https://www.hackster.io/mansurkamsur/upload-sketch-arduino-over-bluetooth-using-android-f1ce55} permite cargar código a la placa Arduino por Bluetooth o USB.
	
\end{itemize}