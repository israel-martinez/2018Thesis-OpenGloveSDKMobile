\section{Resultados obtenidos}
Esta sección se concluye sobre los resultados obtenidos sobre el desarrollo de software y los resultados de las pruebas realizadas en el presente trabajo.

\subsection{Desarrollo de software}
El desarrollo de software de este proyecto tiene como producto final el SDK para dispositivos móviles, el cual se compone de las APIs de alto nivel, la aplicación de configuración, pruebas de concepto y toda la documentación generada durante el proyecto. Estas APIs desarrolladas en Java y C\# dan soporte a la gestión de dispositivos Bluetooth, los actuadores, sensores de flexibilidad  y sensor de rastreo IMU. Gracias a esto es posible hacer uso de OpenGlove en ambientes de Realidad Virtual, Aumentada o Mixta en dispositivos móviles, con tecnologías que sean compatibles con los lenguajes de programación ya mencionados.

\subsection{Resultados de las pruebas}
	
	Los resultados de las pruebas de rendimiento son aceptables dado que no superan el umbral de latencia máximo que las personas pueden detectar. Esto puede ser mejorado utilizando dispositivos con mayores prestaciones que el utilizado en las pruebas de rendimiento, por tanto, las especificaciones de Hardware del dispositivo móvil utilizado, se plantean como requerimientos mínimos recomendados para utilizar OpenGlove en Android, considerando siempre disponer de prestaciones superiores de acuerdo a los requerimientos mínimos para las aplicaciones de Realidad Virtual, Aumentada o Mixta.
%	\subsubsection{Tiempo de respuesta}
%	\subsubsection{Líneas de código}