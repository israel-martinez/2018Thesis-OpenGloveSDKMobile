\section{Resumen}
OpenGlove es un proyecto Open Source\footnote{ ``Open source o código abierto es el término empleado al software distribuido bajo una licencia que permite al usuario acceso al código fuente. Este tipo de licencia posibilita el estudio y la modificación del software con total libertad. Además, su redistribución está permitida siempre y cuando esta posibilidad vaya en concordancia con los términos de licencia bajo la que se adquiere el software" \citep{open-source}.}  que permite la retroalimenación vibrotáctil y comunicación bidireccional entre el guante y las aplicaciones mediante el uso del protocolo WebSocket. Como se ha podido ver, existen diversas alternativas en el mercado de dispositivos hápticos, siendo OpenGlove la alternativa Open Source que no requiere de costos altos de licencias ni del guante,  el cual es desarrollado según las necesidades específicas requeridas. Para lograr que la comunidad de desarrolladores de VR/AR/MR pueda integrar OpenGlove en entornos móviles y realizar una fácil configuración de ellos, se desarrollará un SDK que incluye las APIs de alto nivel necesarias para el desarrollo en las plataformas móviles, la aplicación de configuración y la documentación de uso de las APIs en pruebas de concepto. Es importante señalar la importancia de las pruebas de rendimiento sobre la solución a desarrollar. esto toma relevancia cuando se busca disminuir la latencia de los dispositivos conectados. Esto se debe considerar para brindar una excelente experiencia en entornos virtuales sin que se presenten retardos perceptibles. 

%El creciente mercado y comunidad de desarrolladores de VR, AR y MR, favorece la adopción de estas tecnologías en el tiempo.