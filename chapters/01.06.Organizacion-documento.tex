\section{Organización del documento}

% cuidado con la palabra implementar (es implementar o desarrollar)

El presente documento posee seis capítulos que abarcan la totalidad de este proyecto de desarrollo en sus distintas fases. El Capítulo 1, Introducción, incluye los antecedentes y motivación del proyecto, presentando el problema y solución propuesta junto con los objetivos necesarios para concretarla. El Capítulo 2, Marco teórico, introduce los conceptos relevantes que permitirán una mejor compresión del problema junto a su solución. También se incluye el estado del arte, el cual detalla las soluciones alternativas actuales del problema planteado, detallando y comparándolas entre ellas. En el Capítulo 3, Análisis, se realiza un análisis sobre el desarrollo del SDK analizando los componentes de hardware y software disponibles, generando prototipos hasta que se genere el producto esperado. Luego en el Capítulo 4, Diseño e implementación, se diseña la solución a nivel de mockups  o maquetas, también a nivel arquitectural, comportamiento y de protocolos de comunicación. Posteriormente se implementan para lograr la solución diseñada. El Capítulo 5, Evaluación técnica, se realiza una evaluación del software desarrollado para establecer cuáles son los tiempos de respuesta (latencia) presentes y la comparativa con la versión de escritorio. Además se considera otro factor importante respecto a la solución , esto es, la eficiencia energética de la misma. Por último,  en el Capítulo 6, Conclusiones, se detallan los objetivos cumplidos, también los resultados obtenidos con la solución, los alcances y limitaciones de la misma, posibles mejoras para trabajos futuros y observaciones finales pertinentes.