\chapter{Introducci\'on}
\label{cap:introduccion}

\section{Antecedentes y motivaci\'on}
\label{intro:motivacion}
En 1981 el Sr. Marcelo Pardo Brown es contratado por la Universidad para crear el Departamento de Ingeniería Informática y la carrera de Ingeniería Civil en Informática. Oficialmente, el Departamento fue creado mediante el decreto 286 del 7 de mayo de 1982 y su primer director fue el Sr. Pardo. Ese mismo año, el Departamento asume la tutela de la carrera de Ingeniería de Ejecución en Computación e Informática \citep{Codishetal2000}.

\section{Descripci\'on del problema}
\label{intro:problema}
La carrera de Ingeniería Civil en Informática se crearía oficialmente por Resolución 2324 de 1983.  Los primeros Ingenieros Civiles en Informática de la USACH comenzaron a titularse en el año 1987.

La creación del Departamento permitió entre otras cosas la implementación de un plan de contratación de profesores Jornada Completa y la elaboración de un plan de equipamiento computacional dedicado a las tareas académicas del Departamento. Hasta 1983, los alumnos realizaban sus tareas computacionales usando exclusivamente los recursos Computacionales de SECOM, que en ese entonces consistían principalmente en un computador IBM 370/145 con 256 Kilobytes de memoria. A mediados de los 80, el Departamento adquirió un computador VAX 730 que tenía 2 Megabytes de memoria RAM con el sistema operativo VMS. Además, se habilitó un primer centro de operaciones computacionales y las salas de terminales para uso exclusivo de los alumnos del Departamento.


\section{Soluci\'on propuesta}
\label{intro:solucion}
Físicamente, el Departamento permaneció en las dependencias del Departamento de Ingeniería Industrial hasta 1988, cuando fue trasladado para ocupar lo que fuera el Pabellón de Forja de la Escuela de Artes y Oficios, edificio que es monumento histórico.

Las nuevas dependencias incluían seis salas de laboratorio para docencia, tres laboratorios de investigación para uso de memoristas, cuatro salas de clases, una biblioteca especializada, un centro de operaciones más amplio y apto para las nuevas necesidades tecnológicas del Departamento, oficinas para profesores, administrativos y secretarias docentes para cada carrera. También tuvo baños para uso exclusivo de los miembros del Departamento.

Sin duda este fue el comienzo de una nueva etapa en la vida del Departamento. Su imagen al interior de la Facultad y de la Universidad creció y se potenció con la participación de algunos académicos en proyectos institucionales, como la Dirección de SEGIC y la instalación de una red de fibra óptica en todo el campus universitario.


\section{Objetivos y alcance del proyecto}
\label{intro:objetivos}

\subsection{Objetivo general}
	Objectivo general de la tesis.

\subsection{Objetivos espec\'ificos}
\begin{enumerate}
	\item Objetivo espec\'ifico 1.
	\item Objetivo espec\'ifico 2.
	\item Objetivo espec\'ifico 3.
\end{enumerate}

\subsection{Alcances}
En el año 1993 se creó mediante Resolución 1662 la Prosecución de Estudios conducentes al título de Ingeniero Civil Informático, modalidad vespertina. Este programa de formación está dirigido a profesionales titulados de carreras de Ingeniería de Ejecución del área.

Adicionalmente, un proyecto de equipamiento mayor permitió la compra de un Super-Computador Silicon Graphics para procesamiento paralelo, único en ese momento en el país. Aparece entonces la necesidad de crear tres laboratorios de investigación en las áreas de Procesamiento Paralelo y Optimización, Sistemas Colaborativos y Robótica. La creación de estos laboratorios y la creciente implicación de alumnos como ayudantes de investigación, permitieron abordar proyectos señeros de Asistencia Técnica, como la elaboración de un manual de entrenamiento para el avión Pillán de la Fuerza Aérea de Chile, basado en tecnologías de realidad virtual.

\section{Metodolog\'ia y herramientas utilizadas}
\label{intro:metodologia}

\subsection{Metodolog\'ia}
Se creó el primer programa de titulación especial para ex-alumnos y la carrera de Ingeniería de Ejecución en Computación e Informática modalidad vespertina.

\subsection{Herramientas de desarrollo}
El Departamento comenzó a potenciar sus actividades de investigación mediante la creación del Magíster en Ingeniería Informática (Resolución 386 de 1994), y la implantación de un plan de perfeccionamiento para los académicos conducente a la obtención del grado de Doctor. El plan además consideraba la contratación de nuevos académicos con dicho grado. Es así, como a finales de los 90, el Departamento contaba con nueve doctores jornada completa y otros dos en perfeccionamiento.

\section{Organizaci\'on del documento}
\label{intro:organizacion}
Contar con este grupo de académicos con grado de doctor permitió alcanzar una masa crítica que, junto con los resultados exitosos del Magíster en Ingeniería Informática, motivó al Departamento a trabajar en la elaboración del Programa de Doctorado en Ciencias de la Ingeniería mención Informática. Este programa fue creado por Resolución 6104 del 2000. Ese mismo año el Departamento organizó con éxito las Jornadas Chilenas de Computación, punto de reunión de los académicos nacionales y de los alumnos de Informática del país y que contó además con la participación de destacados investigadores de orden mundial.