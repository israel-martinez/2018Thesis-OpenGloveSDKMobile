\section{Levantamiento de requisitos de software}

\subsection{Antecedentes}
%debería incluirse los requerimientos de las anteriores 2 tesis (en lo que respecta al software) debe especificarse
% trabajo relacionado al SDK alto nivel (extensión de software)
% trabajo relacionado al extensión de OpenGlove a nivel de Hardware y Software
En los trabajos anteriores relacionados a OpenGlove, realizados por \cite{tesis-monsalve-rodrigo}, \cite{tesis-meneses-sebastian} y \cite{tesis-cerda-rodrigo} se ha logrado establecer las bases de hardware y software para dar soporte a la retroalimentación vibrotáctil y el seguimiento de las manos. En el trabajo de \cite{tesis-meneses-sebastian}, se realizó una extensión de sofware a OpenGlove, desarrollando un SDK de alto nivel para la retroalimentación vibrotáctil. Por otra parte, el trabajo de \cite{tesis-cerda-rodrigo}, se realizó una extensión de software y hardware para la captura de movimientos de la mano. En base a estos dos últimos se desprenden las diferentes funcionalidades y soporte que se debe lograr a nivel de software, para dar soporte a los actuadores, sensores de flexibilidad e IMU. Actualmente OpenGlove cuenta con un SDK para Windows, el cual incluye el sofware de configuración y cuatro APIs de alto nivel para los lenguajes C\#, Java, C++ y JavaScript. El SDK fue desarrollado en el IDE Visual Studio.

%Los dispositivos móviles suelen tener menos capacidades de cálculo, memoria y batería, por lo que se esperan tener latencias mayores a las presentadas en los trabajos antes mencionados. El SDK para dispositivos móviles


\subsection{Requisitos}
% ¿Requisitos o Requerimientos? http://investigacionit.com.ar/es/requisitos-o-requerimientos/
% Especificación de Requisitos según el estándar de IEEE 830 https://www.fdi.ucm.es/profesor/gmendez/docs/is0809/ieee830.pdf

Al realizar un análisis de los antecedentes y el problema planteado en el Capítulo 1, se capturan los siguientes requisitos funcionales del software.

%%%%%%%%%%%%%%%%%%%%%%%%%%%%%%%%%%%%%%%%%%%%
% https://www.tablesgenerator.com/
%\caption[Requisitos funcionales de software]{Requisitos funcionales de software \\ Fuente: Elaboración propia (2018)}
%\label{table:RF}\\

%{START} .../latex-2018-OpenGlove-SDK/tables-generator/functional-requeriments

% Please add the following required packages to your document preamble:
% \usepackage{longtable}
% Note: It may be necessary to compile the document several times to get a multi-page table to line up properly
\begin{longtable}[c]{|l|l|l|l|}
\caption[Requisitos funcionales de software]{Requisitos funcionales de software \\ Fuente: Elaboración propia (2018)}
\label{table:RF}\\
\hline
ID & Síntesis del Requisito & Descripción & Origen \\ \hline
\endfirsthead
%
\endhead
%
RF001 & \begin{tabular}[c]{@{}l@{}}El sistema debe permitir\\ al usuario ver los\\ dispositivos OpenGlove\\ emparejados\end{tabular} & \begin{tabular}[c]{@{}l@{}}El sistema debe ofrecer una \\ recopilación de los dispositivos \\ OpenGlove emparejados con\\ el dispositivo móvil, su estado de \\ conexión actual y la dirección \\ MAC del mismo.\end{tabular} & \begin{tabular}[c]{@{}l@{}}Inicio,\\ modificado de\\ Meneses (2016)\end{tabular} \\ \hline
RF002 & \begin{tabular}[c]{@{}l@{}}El sistema debe permitir\\ al usuario definir la\\ configuración de\\ hardware de cada guante\end{tabular} & \begin{tabular}[c]{@{}l@{}}Cada placa LilyPad posee pines \\ programables, en los cuales el \\ usuario puede conectar actuadores,\\ sensores de flexibilidad e IMU \\ para crear su propio OpenGlove. \\ Es necesario que se establezca\\ esta configuración en el sistema\\ para cada guante, ya que de ella \\ depende la activación de actuadores\\ y la lectura de datos de los flexores\\ e IMU. Se debe diferenciar pines\\ digitales y análogos de la placa\\ Arduino\end{tabular} & \begin{tabular}[c]{@{}l@{}}Inicio,\\ modificado de\\ Meneses (2016)\\ y RF001 de\\ Cerda (2017)\end{tabular} \\ \hline
RF003 & \begin{tabular}[c]{@{}l@{}}El sistema debe permitir\\ al usuario guardar una\\ configuración de\\ hardware\end{tabular} & \begin{tabular}[c]{@{}l@{}}Al crear un nuevo perfil de hardware\\ para un guante, el sistema debe contar\\ con un mecanismo para la persistencia\\ de esta configuración.\end{tabular} & \begin{tabular}[c]{@{}l@{}}Inicio,\\ Meneses (2016)\end{tabular} \\ \hline
RF004 & \begin{tabular}[c]{@{}l@{}}El sistema debe permitir\\ al usuario abrir una\\ configuración de\\ hardware previamente\\ almacenada por el\\ sistema\end{tabular} & \begin{tabular}[c]{@{}l@{}}Una vez guardada una configuración,\\ esta debe ser reconocible por el \\ sistema para su uso posterior en otro\\ guante.\end{tabular} & \begin{tabular}[c]{@{}l@{}}Inicio,\\ Meneses (2016)\end{tabular} \\ \hline
RF005 & \begin{tabular}[c]{@{}l@{}}El sistema debe permitir\\ al usuario definir la\\ configuración de \\ actuadores de cada\\ guante\end{tabular} & \begin{tabular}[c]{@{}l@{}}Dependiente de la configuración de\\ hardware, la configuración de\\ actuadores es una representación de\\ la distribución física de los actuadores\\ LilyPad en una  mano virtual. Esta\\ representación permite establecer\\ mapeos región-actuador usables en\\ una API de alto nivel. Se debe ofrecer\\ al usuario una solución gráfica que\\ permita ordenar la posición de los\\ actuadores en una representación de\\ la mano. También debe permitir la\\ posibilidad de agregar su propia\\ imagen que represente el mapeo.\end{tabular} & \begin{tabular}[c]{@{}l@{}}Inicio, \\ modificado de\\ Meneses (2016)\end{tabular} \\ \hline
RF006 & \begin{tabular}[c]{@{}l@{}}El sistema debe permitir\\ al usuario guardar una\\ configuración de \\ actuadores\end{tabular} & \begin{tabular}[c]{@{}l@{}}Al crear un nuevo perfil de actuadores\\ para un guante, el sistema debe contar\\ con un mecanismo para la persistencia\\ de estaconfiguración.\end{tabular} & \begin{tabular}[c]{@{}l@{}}Inicio,\\ Meneses (2016)\end{tabular} \\ \hline
RF007 & \begin{tabular}[c]{@{}l@{}}El sistema debe permitir\\ al usuario abrir una \\ configuración de\\ actuadores previamente\\ almacenada por el\\ sistema\end{tabular} & \begin{tabular}[c]{@{}l@{}}Una vez guardada una configuración\\ de actuadores, esta debe ser\\ reconocible por el sistema para su uso\\ posterior en otro guante.\end{tabular} & \begin{tabular}[c]{@{}l@{}}Inicio,\\ Meneses (2016)\end{tabular} \\ \hline
RF008 & \begin{tabular}[c]{@{}l@{}}El sistema debe permitir\\ al usuario establecer\\ conexión con un \\ dispositivo OpenGlove\\ emparejado\end{tabular} & \begin{tabular}[c]{@{}l@{}}Una vez emparejado un guante\\ OpenGlove, el sistema debe permitir\\ que el usuario inicie la conexión\\ Bluetooth. No es necesario que el\\ usuario especifique la dirección\\ MAC del guante.\end{tabular} & \begin{tabular}[c]{@{}l@{}}Inicio,\\ modificado de \\ Meneses (2016)\end{tabular} \\ \hline
RF009 & \begin{tabular}[c]{@{}l@{}}El sistema debe permitir\\ al usuario activar una \\ región de un guante con\\ intensidad a voluntad\end{tabular} & \begin{tabular}[c]{@{}l@{}}El sistema debe exponer al usuario\\ una sección que le permita activar\\ una región de un guante con la\\ intensidad (entre 0 y 255) que él\\ desee para probar el hardware. Esta\\ región esta predefinida y debe ser\\ independiente de la configuración\\ de hardware (actuadores) presente\\ en el guante. Esta función debe\\ estar disponible para uno o varios\\ actuadores en un guante.\end{tabular} & \begin{tabular}[c]{@{}l@{}}Inicio,\\ modificado de\\ Meneses (2016)\end{tabular} \\ \hline
RF010 & \begin{tabular}[c]{@{}l@{}}El sistema debe permitir\\ al usuario operar con\\ distintas implementa-\\ ciones de OpenGlove\end{tabular} & \begin{tabular}[c]{@{}l@{}}Al momento de crear un nuevo\\ perfil de hardware, el sistema\\ debe proveer un mecanismo para\\ que el usuario genere su propia\\ placa, lo que se traduce en un\\ nombre y una cantidad de pines\\ para poder usarla en su\\ configuración.\end{tabular} & \begin{tabular}[c]{@{}l@{}}Inicio,\\ Meneses (2016)\end{tabular} \\ \hline
RF011 & \begin{tabular}[c]{@{}l@{}}El sistema actual debe\\ permitir al usuario\\ guardar y cargar una\\ configuración de\\ hardware incluyendo\\ los flexores\end{tabular} & \begin{tabular}[c]{@{}l@{}}El SDK debe ser capaz de guardar\\ y cargar una configuración de\\ hardware, compuesta por  actuadores\\ y/o sensores de flexibilidad.\end{tabular} & \begin{tabular}[c]{@{}l@{}}Inicio,\\ modificado de\\ Cerda (2017)\end{tabular} \\ \hline
RF012 & \begin{tabular}[c]{@{}l@{}}El sistema debe permitir\\ al usuario crear una\\ configuración de los\\ sensores de flexibilidad\\ y del sensor de rastreo\\ IMU\end{tabular} & \begin{tabular}[c]{@{}l@{}}El software debe ser capaz de dar\\ soporte para la creación de nuevas\\ configuraciones de los sensores de\\ flexibilidad y el IMU.\end{tabular} & \begin{tabular}[c]{@{}l@{}}Inicio,\\ modificado de \\ Cerda (2017)\end{tabular} \\ \hline
RF013 & \begin{tabular}[c]{@{}l@{}}El sistema debe permitir\\ al usuario seleccionar\\ un sensor de flexibilidad\\ y  relacionarlo a una\\ región del guante\end{tabular} & \begin{tabular}[c]{@{}l@{}}La configuración correspondiente\\ a los sensores de flexibilidad, debe\\ ser capaz de seleccionar una región\\ del guante y relacionarlo con un\\ sensor de flexibilidad.\end{tabular} & \begin{tabular}[c]{@{}l@{}}Inicio,\\ Cerda (2017)\end{tabular} \\ \hline
RF014 & \begin{tabular}[c]{@{}l@{}}El sistema debe permitir\\ al usuario eliminar un\\ sensor de flexibilidad\\ de una región del guante\end{tabular} & \begin{tabular}[c]{@{}l@{}}La configuración de los sensores de\\ flexibilidad debe ser capaz de\\ eliminar un flexor de una región del\\ guante, de tal manera que la región\\ quede libre y el flexor pueda asignarse\\ a una nueva región.\end{tabular} & \begin{tabular}[c]{@{}l@{}}Inicio,\\ Cerda (2017)\end{tabular} \\ \hline
RF015 & \begin{tabular}[c]{@{}l@{}}El sistema debe enviar\\ los datos provenientes\\ de los sensores de\\ flexibilidad\\ automáticamente\end{tabular} & \begin{tabular}[c]{@{}l@{}}Cuando un sensor de flexibilidad es\\ asignado a una región del guante, el\\ sistema debe ser capaz de transmitir\\ los datos de dicho sensor de manera\\ automática, especificando el tipo de\\ dato, región y el valor leído.\end{tabular} & \begin{tabular}[c]{@{}l@{}}Inicio,\\ Cerda (2017)\end{tabular} \\ \hline
RF016 & \begin{tabular}[c]{@{}l@{}}El sistema debe parar\\ de enviar los datos \\ provenientes de los \\ sensores de flexibilidad\\ automáticamente\end{tabular} & \begin{tabular}[c]{@{}l@{}}Cuando un sensor de flexibilidad es\\ eliminado de una región del guante,\\ el sistema debe ser capaz de parar la\\ transmisión de datos de  dicho flexor\\  automáticamente.\end{tabular} & \begin{tabular}[c]{@{}l@{}}Inicio,\\ Cerda (2017)\end{tabular} \\ \hline
RF017 & \begin{tabular}[c]{@{}l@{}}El sistema debe permitir\\ al usuario definir un\\ threshold al momento\\ de enviar datos de los \\ sensores de flexibilidad\end{tabular} & \begin{tabular}[c]{@{}l@{}}El guante enviará el dato de un flexor,\\ solo si este dato posee una diferencia\\ mayor o igual al valor definido como\\ threshold, con respecto al último valor\\ enviado por dicho flexor.\end{tabular} & \begin{tabular}[c]{@{}l@{}}Inicio,\\ Cerda (2017)\end{tabular} \\ \hline
RF018 & \begin{tabular}[c]{@{}l@{}}El sistema debe permitir\\ al usuario la opción de \\ calibrar los sensores de \\ flexibilidad\end{tabular} & \begin{tabular}[c]{@{}l@{}}Los datos provenientes de los sensores\\ de flexibilidad deben ser calibrados\\ con respecto al máximo y mínimo\\ valor leído de la articulación de un\\ dedo.\end{tabular} & \begin{tabular}[c]{@{}l@{}}Inicio,\\ Cerda (2017)\end{tabular} \\ \hline
RF019 & \begin{tabular}[c]{@{}l@{}}El sistema debe permitir\\ al usuario probar los\\ sensores de flexibilidad\end{tabular} & \begin{tabular}[c]{@{}l@{}}Luego de asignar uno o más sensores\\ de  flexibilidad a una región del\\ guante, se debe habilitar un botón que\\ permita probar si la  configuración es\\ correcta, visualizando el valor\\ entregado por cada flexor en dicha\\ región.\end{tabular} & \begin{tabular}[c]{@{}l@{}}Inicio,\\ Cerda (2017)\end{tabular} \\ \hline
RF020 & \begin{tabular}[c]{@{}l@{}}El usuario debe permitir\\ al usuario poder obtener\\ datos desde un sensor de\\ rastreo IMU\end{tabular} & \begin{tabular}[c]{@{}l@{}}La configuración correspondiente al\\ sensor de rastreo IMU, debe ser\\ capaz de activar o desactivar el envío\\ de datos de ésta.\end{tabular} & \begin{tabular}[c]{@{}l@{}}Inicio,\\ Cerda (2017)\end{tabular} \\ \hline
RF021 & \begin{tabular}[c]{@{}l@{}}El sistema debe permitir\\ al usuario poder obtener\\ datos crudos o \\ procesados desde\\ el sensor de rastreo IMU\end{tabular} & \begin{tabular}[c]{@{}l@{}}La configuración del sensor de\\ rastreo IMU, debe ser capaz de definir\\ si los datos enviados por ésta son\\ procesados o no.\end{tabular} & \begin{tabular}[c]{@{}l@{}}Inicio,\\ Cerda (2017)\end{tabular} \\ \hline
RF022 & \begin{tabular}[c]{@{}l@{}}El sistema debe permitir\\ al usuario poder probar\\ el sensor de rastreo IMU\end{tabular} & \begin{tabular}[c]{@{}l@{}}Luego de activar el envío de datos del\\ sensor de rastreo IMU, se debe activar\\ un botón que active la visualización de\\ todos los datos entregados por el sensor.\end{tabular} & \begin{tabular}[c]{@{}l@{}}Inicio,\\ Cerda (2017)\end{tabular} \\ \hline
RF023 & \begin{tabular}[c]{@{}l@{}}El sistema debe permitir\\ al usuario calibrar el\\ sensor de rastreo IMU\end{tabular} & \begin{tabular}[c]{@{}l@{}}Los datos provenientes del IMU deben\\ ser calibrados bajo una posición de\\ referencia,  para poder determinar la\\ posición y orientación de la mano.\end{tabular} & Nuevo \\ \hline
\end{longtable}
%{END} tables-generator/functional-requeriments
%%%%%%%%%%%%%%%%%%%%%%%%%%%%%%%%%%%%%%%%%%%%















La Tabla \ref{table:RNF}, muestra los requisitos no funcionales obtenidos mediante las reuniones con el profesor guía. Al igual que lo señalado por \cite{tesis-monsalve-rodrigo}, de esta lista de requisitos para el SDK, se debe dar especial importancia a RNF004, el cual se refiere al umbral de tiempo que no se debe superar para ofrecer la retroalimentación táctil. Experimentos enfocados en medir los los tiempos en que una persona puede percibir inconsistencias entre una interacción táctil con un objeto virtual y un estímulo táctil recibido, obtienen los siguientes resultados: 54 ms para retroalimentación kinestésica \citep{latency-haptic-kinestesic-2004}, 60 ms en estudios de percepción \citep{latency-haptic-perception-2009} y 60 ms simulaciones quirúrgicas simulaciones \citep{latency-visual-haptic-2015}.


%%%%%%%%%%%%%%%%%%%%%%%%%%%%%%%%%%%%%%%%%%%%
% https://www.tablesgenerator.com/
%\caption[Requisitos no funcionales de software]{Requisitos  no funcionales de software \\ Fuente: Elaboración propia (2018)}
%\label{table:RNF}

\begin{table}[H]
\caption[Requisitos no funcionales de software]{Requisitos  no funcionales de software \\ Fuente: Elaboración propia (2018)}
\label{table:RNF}
\begin{tabular}{|l|l|l|l|}
\hline
ID     & Síntesis del requisito                                                    & Descripción                                                                                                                                                                                                                      & Origen \\ \hline
RNF001 & Sistema multiplataforma                                                   & \begin{tabular}[c]{@{}l@{}}El software a desarrollar debe ser ejecutado\\ inicialmente en el sistema operativo Android,\\ sin embargo, el proyecto en un futuro debe \\ poder dar soporte al sistema operativo iOS.\end{tabular} & Inicio \\ \hline
RNF002 & \begin{tabular}[c]{@{}l@{}}Soportar comunicación\\ Bluetooth\end{tabular} & \begin{tabular}[c]{@{}l@{}}El sistema debe ser capaz de comunicarse\\ con los dispositivos OpenGlove mediante \\ Bluetooth Clásico.\end{tabular}                                                                                 & Inicio \\ \hline
RNF003 & Interoperabilidad                                                         & \begin{tabular}[c]{@{}l@{}}El SDK debe permitir la interoperabilidad\\ entre los lenguajes de programación C\# y\\ Java.\end{tabular}                                                                                            & Inicio \\ \hline
RNF004 & Umbral latencia aceptada                                                  & \begin{tabular}[c]{@{}l@{}}El SDK no debe superar el umbral de 60 ms\\ de latencia perceptible por el usuario.\end{tabular}                                                                                                      & Inicio \\ \hline
\end{tabular}
\end{table}

%{START} .../2018Thesis-OpenGloveSDKMobile/tables-generator/no-functional-requeriments



%{END} .../2018Thesis-OpenGloveSDKMobile/tables-generator/no-functional-requeriments
