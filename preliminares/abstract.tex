\resumenCastellano{
OpenGlove es un dispositivo diseñado en el Departamento de Ingeniería Informática por el grupo de investigación InTeracTion perteneciente a la Universidad de Santiago de Chile. El guante permite la retroalimentación vibrotáctil o \textit{haptic feedback} cuando se interactúa con objetos en ambientes de realidad virtual (VR), aumentada (AR) o mixta (MR). Esta retroalimentación es generada a través de la vibración de motores distribuidos en distintas partes del guante, distribución que depende de los requerimientos del usuario o la aplicación. El guante también sensores de flexibilidad para la captura del movimiento de los dedos y una unidad de medición inercial o IMU(Inertial Measurement Unit) para capturar la orientación de la mano.

%OpenGlove fue concebido para que pueda ser utilizado en conjunto con otros dispositivos, tales como Oculus Rift, Kinect y Leap Motion. OpenGlove actualmente soporta un servicio bluetooth en Windows para la conexión de dispositivos. El guante adicionalmente soporta el uso de sensores de flexibilidad  y una unidad de medición inercial o IMU (Inertial Measurement Unit).

Si bien es posible tener una inmersión en estos ambientes,  esta es parcial, dejando de lado el sentido del tacto. El estado actual de OpenGlove no permite el uso del mismo en comunidades de desarrollo de VR, AR y MR en entornos móviles como Android e iOS. Esto limita la portabilidad y desacople del sistema operativo Windows. Por ello lo que se propone es dar soporte a OpenGlove en dispositivos móviles enriqueciendo la interacción en los entornos ya mencionados.
%Como
Esto se realizó mediante el desarrollo de un SDK para dispositivos móviles, el cual permite la conexión por Bluetooth de varias instancias de OpenGlove, como también la  configuración de los mismos mediante la carga de perfiles utilizando una aplicación de configuración. Esta aplicación expone un servicio, el cual es utilizado mediante las APIs en Java y C\#. Se realizaron evaluaciones de rendimiento y pruebas de concepto utilizando el sistema propuesto.

% el qué, el cómo, para qué y por qué

%OpenGlove es un dispositivo que permite la retroalimentación vibrotáctil en ambientes de realidad virtual. Fue diseñado en la Universidad de Santiago de Chile, su diseño es flexible dado que no se limita a una ubicación en especial de los dispositivos vibrotáctiles. Los prototipos realizados utilizan motores que generan vibraciones cuando se interactúa  con objetos en un entorno virtual o de realidad aumentada. OpenGlove se ha pensado para que pueda ser utilizado en conjunto con otros dispositivos, como Oculus Rift, Kinect y Leap Motion. Por otra parte, los prototipos de OpenGlove utilizan motores configurables con diferentes niveles de potencia, lo que permite la respuesta vibrotáctil en distintas áreas de la mano, lo cual puede ser utilizado para representar la sensación de tocar objetos en entornos virtuales, como también, recibir retroalimentación que represente impacto, el cual sería útil en un juego de boxeo por ejemplo.   Actualmente, el proyecto OpenGlove considera una arquitectura que involucra el guante que utiliza una placa Arduino LilyPad, con el cual puede ser configurado y utilizado mediante el SDK y la API de bajo nivel. Luego en un nivel superior de la arquitectura, existe un servicio que utiliza SOAP y REST para exponer los servicios mediante bluetooth en Windows y hacer uso del guante, como también el configurar parámetros de los componentes del guante. Finalmente, las  APIs de alto nivel para lenguajes de programación como C\#, C++, Java, y JavaScript implementadas el 2016, permiten desacoplamiento y abstraen  la complejidad en el uso de instancias y configuración de OpenGlove. El estado actual de OpenGlove no permite el uso del mismo en comunidades de desarrollo de realidad virtual móvil en Android, limitando con ello la portabilidad y desacople del sistema operativo Windows. Por ello lo que se propone es dar soporte a OpenGlove en dispositivos móviles. Esto se realizará mediante el desarrollo de una aplicación nativa en Android, que permitirá levantar un servicio bluetooth para el uso de varias instancias de OpenGlove, como también la configuración de los guantes, mediante la carga de perfiles. También se desarrollará un SDK para facilitar los futuros desarrollos en dispositivos móviles. Se realizarán evaluaciones de rendimiento y pruebas de concepto utilizando el sistema propuesto.

\vspace*{0.5cm}
\KeywordsES{ Haptic Feedback; Virtual Reality (VR); Augmented Reality (AR); Mixed Reality (MR); OpenGlove; SDK; API}
}

%\newpage

%\resumenIngles{
%Contrary to popular belief, Lorem Ipsum is not simply random text. It has roots in a piece of classical Latin literature from 45 BC, making it over 2000 years old. Richard McClintock, a Latin professor at Hampden-Sydney College in Virginia, looked up one of the more obscure Latin words, consectetur, from a Lorem Ipsum passage, and going through the cites of the word in classical literature, discovered the undoubtable source. Lorem Ipsum comes from sections 1.10.32 and 1.10.33 of "de Finibus Bonorum et Malorum" (The Extremes of Good and Evil) by Cicero, written in 45 BC. This book is a treatise on the theory of ethics, very popular during the Renaissance. The first line of Lorem Ipsum, "Lorem ipsum dolor sit amet..", comes from a line in section 1.10.32.

%The standard chunk of Lorem Ipsum used since the 1500s is reproduced below for those interested. Sections 1.10.32 and 1.10.33 from "de Finibus Bonorum et Malorum" by Cicero are also reproduced in their exact original form, accompanied by English versions from the 1914 translation by H. Rackham.

%\vspace*{0.5cm}
%\KeywordsEN{Key; words}
%}
